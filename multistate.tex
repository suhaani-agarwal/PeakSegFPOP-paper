\documentclass{article}

% use Times
\usepackage{times}
% For figures
\usepackage{graphicx} % more modern
%\usepackage{epsfig} % less modern
\usepackage{subfigure} 

% For citations
\usepackage{natbib}

% For algorithms
\usepackage{algorithm}
\usepackage{algorithmic}

% As of 2011, we use the hyperref package to produce hyperlinks in the
% resulting PDF.  If this breaks your system, please commend out the
% following usepackage line and replace \usepackage{icml2016} with
% \usepackage[nohyperref]{icml2016} above.
\usepackage{hyperref}
\usepackage{xcolor}
\definecolor{Ckt}{HTML}{E41A1C}
\definecolor{Min}{HTML}{4D4D4D}%grey30
%{B3B3B3}%grey70
\definecolor{MinMore}{HTML}{377EB8}
\definecolor{Data}{HTML}{984EA3}

% Packages hyperref and algorithmic misbehave sometimes.  We can fix
% this with the following command.
\newcommand{\theHalgorithm}{\arabic{algorithm}}

% Employ the following version of the ``usepackage'' statement for
% submitting the draft version of the paper for review.  This will set
% the note in the first column to ``Under review.  Do not distribute.''
%\usepackage{icml2016} 
\usepackage{fullpage}

% Employ this version of the ``usepackage'' statement after the paper has
% been accepted, when creating the final version.  This will set the
% note in the first column to ``Proceedings of the...''
%\usepackage[accepted]{icml2016}

\usepackage{tikz}
\usetikzlibrary{arrows}
\usepackage{amssymb,amsmath}
\usepackage{natbib}
\usepackage{amsthm}
\newtheorem{proposition}{Proposition}
\newtheorem{lemma}{Lemma}
\newtheorem{theorem}{Theorem}
\newtheorem{definition}{Definition}
\DeclareMathOperator*{\argmin}{arg\,min}
\DeclareMathOperator*{\sign}{sign}
\DeclareMathOperator*{\Lik}{Lik}
\DeclareMathOperator*{\Peaks}{Peaks}
\DeclareMathOperator*{\HotSpots}{HotSpots}
\newcommand{\Cost}{\text{Cost}}
\usepackage{stfloats}
\DeclareMathOperator*{\Diag}{Diag}
\DeclareMathOperator*{\TPR}{TPR}
\DeclareMathOperator*{\Segments}{Segments}
\DeclareMathOperator*{\Changes}{Changes}
\DeclareMathOperator*{\FPR}{FPR}
\DeclareMathOperator*{\argmax}{arg\,max}
\DeclareMathOperator*{\maximize}{maximize}
\DeclareMathOperator*{\minimize}{minimize}
\newcommand{\ZZ}{\mathbb Z}
\newcommand{\NN}{\mathbb N}
\newcommand{\RR}{\mathbb R}



\begin{document}

\title{A constrained changepoint model for neuro data}
\author{
Toby Dylan Hocking (toby.hocking@r-project.org)
}
\maketitle

\begin{abstract}
We propose a three-state model for neuro data.
\end{abstract}

A functional pruning algorithm can solve the following problem. Let
$G=(V,E)$ be a directed graph that represents the model constraints
(note that some edges have penalty 0, so that we only penalize a change if it
ends up in the dec state, resulting in a new spike).
\begin{center}
  \begin{tikzpicture}[->,>=stealth',shorten >=1pt,auto,node distance=3cm,
                    thick,main node/.style={circle,draw}]

  \node[main node] (inc) {inc};
  \node[main node] (flat) [below left of=inc] {flat};
  \node[main node] (dec) [below right of=inc] {dec};

  \path[every node/.style={font=\sffamily\small}]
  (flat) edge [bend left] node {$g_\uparrow, 0$} (inc)
  (inc) edge [bend left] node {$g_\downarrow, \lambda$} (dec)
  (dec) edge [] node {$g_\uparrow, 0$} (inc)
  (flat) edge [] node {$g_\uparrow,\lambda$} (dec)
  (dec) edge [bend left] node {$g_\uparrow,0$} (flat)
  (dec) edge [loop right] node {$g_\uparrow, \lambda$} (dec)
;
\end{tikzpicture}
\end{center}
The vertices $V=\{1,\dots,|V|\}$ can be represented as integers, one
for every distinct state (here $|V|=3$). Each state $s\in V$ has a
function $f_s$ that determines how the mean evolves
$f_s(m_t) = m_{t+1}$ in that state:
\begin{center}
  \begin{tabular}{ccc}
state $s$ &  function $f_s(m)$ & interpretation \\
  \hline
  flat & $m$ & constant after a long time with no spikes\\
dec & $\gamma m$ & decreasing after a spike\\
inc & $m+b$ & increasing before a spike
\end{tabular}
\end{center}
The edges $E=\{1,\dots,|E|\}$ is another set of integers, each of
which represents one of the possible changes between states. Each
edge/change $c\in E$ has corresponding data
$(\underline v_c, \overline v_c, \lambda_c, \bar g_c)$ which specifies
a transition from state $\underline v_c$ to state $\overline v_c$,
with a penalty of $\lambda_c\in\RR_+$, and a constraint function
$\bar g_c:\RR\times\RR\rightarrow\RR$. We enforce the constraint
$\bar g_c[f_{\underline v_c}(m_t), m_{t+1}]= g_c(m_t, m_{t+1})\leq
0$. Note that there are two equivalent ways to write the constraint:
using $\bar g_c$ is more convenient for defining the model (we can use
$g_\downarrow(m_t,m_{t+1})=m_{t+1}-m_t$ for a non-increasing change, and
$g_\uparrow(m_t,m_{t+1})=m_t-m_{t+1}$ for a non-decreasing change), and using
$g_c$ is more convenient for implementing the solver. For our model we
have
\begin{center}
  \begin{tabular}{ccccccc}
  from $\underline v_c$ & to $\overline v_c$ & penalty $\lambda_c$ &
  $\bar g_c$ & $g_c(m_t, m_{t+1})$ & interpretation\\
  \hline
  flat & inc & 0 & $g_\uparrow$ & $m_t-m_{t+1}$ & gradual increase to a spike\\
  flat & dec & $\lambda$ & $g_\uparrow$ & $m_t-m_{t+1}$ & instantaneous spike\\
  dec & dec & $\lambda$ & $g_\uparrow$ & $\gamma m_t-m_{t+1}$ & instantaneous spike\\
  dec & flat & 0 & $g_\uparrow$ & $\gamma m_t-m_{t+1}$ & no activity long after a spike\\
  dec & inc & 0 & $g_\uparrow$ & $\gamma m_t-m_{t+1}$ & gradual increase to a spike\\
  inc & dec & $\lambda$ & $g_\downarrow$ & $m_{t+1}-m_t-b$ & peak of spike
\end{tabular}
\end{center}
In the optimization problem below we
also allow $c=0$, which implies no penalty $\lambda_0=0$, and means no
change:
\begin{align}
  \label{eq:GFPOP_problem}
  \minimize_{
    \substack{
    \mathbf m\in\RR^n,\ \mathbf s\in V^n\\
\mathbf c\in \{0,1,\dots,|E|\}^{n-1}
}
    } &\ \ 
  \sum_{t=1}^n \ell(y_t, m_t) + \sum_{t=1}^{n-1} \lambda_{c_t} \\
  \text{subject to \hskip 0.9cm} &\ \ c_t = 0 \Rightarrow f_{s_t}(m_t) = m_{t+1}
\text{ and } s_t= s_{t+1}
  \nonumber\\
&\ \ c_t \neq 0 \Rightarrow g_{c_t}(m_t, m_{t+1})\leq 0\text{ and }
(s_t,s_{t+1})=(\underline v_{c_t}, \overline v_{c_t}).
\nonumber
\end{align}
If some states are desired at the start or end, then those constraints
$s_1\in \underline S, s_n\in\overline S$ can also be enforced.  To
compute the solution to this optimization problem, we propose the
following dynamic programming algorithm.

Let $C_{s,t}(u)$ be the optimal cost with mean $u$ and state $s$ at
data point $t$. This quantity can be recursively computed using
dynamic programming. The initialization for the first data point is
$ C_{s,1}(u) = \ell(y_1, u)$ for all states $s$. The dynamic
programming update rule for all data points $t>1$ is TODO.

\end{document} 
