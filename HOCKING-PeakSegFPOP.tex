\documentclass{article}

\usepackage{graphicx}
\usepackage[cm]{fullpage}
\usepackage{amssymb,amsmath}
\usepackage{natbib}
\DeclareMathOperator*{\argmin}{arg\,min}
\DeclareMathOperator*{\sign}{sign}
\DeclareMathOperator*{\Lik}{Lik}
\DeclareMathOperator*{\Peaks}{Peaks}
\DeclareMathOperator*{\HotSpots}{HotSpots}
\newcommand{\Cost}{\text{Cost}}
\usepackage{stfloats}
\DeclareMathOperator*{\Diag}{Diag}
\DeclareMathOperator*{\TPR}{TPR}
\DeclareMathOperator*{\Segments}{Segments}
\DeclareMathOperator*{\FPR}{FPR}
\DeclareMathOperator*{\argmax}{arg\,max}
\DeclareMathOperator*{\maximize}{maximize}
\DeclareMathOperator*{\minimize}{minimize}
\newcommand{\ZZ}{\mathbb Z}
\newcommand{\NN}{\mathbb N}
\newcommand{\RR}{\mathbb R}

\begin{document}

\title{A linear time algorithm for peak detection using constrained
  optimal segmentation}

\author{Toby Dylan Hocking}

\maketitle

\section{Introduction}

\citet{FPOP} proposed the Functional Pruning Optimal Partitioning
(FPOP) Algorithm to exactly solve the penalized segmentation
problem. In this paper we use the FPOP algorithm to solve the
penalized version of the PeakSeg problem \citep{PeakSeg}.

\section{Problem setting}

\subsection{Unconstrained model}

We have a genomic data set $\mathbf{z}\in\ZZ_+^B$ of counts on $B$
bases. The unconstrained problem for a positive penalty parameter
$\lambda\in\RR_+$ is
\begin{equation}
  \label{unconstrained}
  \mathbf{\hat m}^\lambda(\mathbf z)  =\ 
  \argmin_{\mathbf m\in\RR^{B}}\ 
  \rho
  %\tag{\textbf{Unconstrained}}
  (\mathbf m, \mathbf z) 
  +\lambda\Segments(\mathbf m),
\end{equation}
where the Poisson loss function is
\begin{equation}\label{eq:rho}
  \rho(\mathbf m, \mathbf z)= \sum_{b=1}^B m_b - z_b \log m_b.
\end{equation} 
The model complexity is the number of piecewise constant segments
\begin{equation}
  \Segments(\mathbf m)=1+\sum_{b=2}^B I(m_{b-1} \neq m_b),
\end{equation}
where $I$ is the indicator function.

Although it is a non-convex optimization problem, the segmentation
$\mathbf{\hat m}^\lambda(\mathbf z)$ can be computed in linear $O(B)$
time using the FPOP algorithm \citep{FPOP}.

We refer to (\ref{unconstrained}) as the ``unconstrained'' model since
$\mathbf{\hat m}^\lambda(\mathbf z)$ can have any sequence of changes
up or down. However for the purposes of peak detection, we are only
interested in segmentations with alternating changes that can be
interpreted as peaks and background \citep{PeakSeg}.

More concretely, we first define the peak indicator at base
$b\in\{2, \dots, B\}$ as
\begin{equation}
  \label{eq:peaks}
  P_b(\mathbf m) = \sum_{b=2}^B \sign( m_{b} - m_{b-1} ),
\end{equation}
where $P_1(\mathbf m)=0$ by convention. $P_b(\mathbf m)$ is the
cumulative sum of signs of changes up to point $b$ in the piecewise
constant vector $\mathbf m$. We define the vector of peak indicators
as
\begin{equation}
  \mathbf
  P[\mathbf m] = \left[
    \begin{array}{ccc}
      P_1(\mathbf m) & \cdots & P_B(\mathbf m)
    \end{array}\right].
\end{equation}

\subsection{PenPeakSeg: penalized constrained segmentation}
\label{sec:constrained}

In general for the unconstrained model $P_b(\mathbf m)\in\ZZ$, which
is problematic since we want to use it as a peak detector with binary
outputs $P_b(\mathbf m)\in \{0, 1\}$. 
For example, if $\mathbf m = \left[\begin{array}{ccccccc}1.1 &
    1.1 & 2 & 2 & 4 & 4 & 3\end{array}\right]$, with two changes up
followed by one change down, then $\mathbf P(\mathbf m) =
\left[\begin{array}{ccccccc}0 & 0 & 1 & 1 & 2 & 2 &
    1 \end{array}\right]$.
Thus we constrain the peak indicator $P_b(\mathbf m)\in\{0, 1\}$,
which results in the constrained problem
\begin{align*}
  \label{PenPeakSeg}
  \mathbf{\tilde m}^\lambda(\mathbf z)  =
  \argmin_{\mathbf m\in\RR^{B}} &\ \ 
    \rho(\mathbf m, \mathbf z) + \lambda\Segments(\mathbf m)
    \tag{\textbf{PenPeakSeg}}
  \\
  \text{such that} &\ \forall b\in\{1, \dots, B\}, \ \ P_b(\mathbf m) \in\{0, 1\}.
\end{align*}
The constraint in the \ref{PenPeakSeg} problem forces the sequence of
changes in the segment means $\mathbf m$ to begin with a positive
change and then alternate: up, down, up, down, ... (and not up, up,
down). Thus the even-numbered segments may be interpreted as peaks
$P_b(\mathbf m)=1$, and the odd-numbered segments may be interpreted
as background $P_b(\mathbf m)=0$.

\section{Algorithm}

\section{Figures}

In this section we are concerned with showing that the new algo can
recover the optimal constrained segmentation for
$\mathbf z = \left[\begin{array}{cccc} 1 & 10 & 14 & 13
\end{array}\right]\in\ZZ_+^4
$. For $s=3$ segments there are only 3 possible segmentations:
$[1][10][14, 13]$, $[1][10, 14][13]$ and $[1, 10][14][13]$. If we use
max-likelihood estimates for each segment mean, then only the last
segmentation obeys the \ref{PeakSeg} model constraints. However the
constrained Dynamic Programming Algorithm of \citet{PeakSeg} does not
recover it.

In the figures below I took the R implementation of PDPA and FPOP from
the FPOP paper, and I ran both on this simple data set. The plot shows
the functions and intervals at each time step. The first column shows
the model ofthe cost before adding a new data point, the second column
shows after adding a new data point but before pruning, and the third
column shows after pruning.

\includegraphics[width=\textwidth]{figure-unconstrained-FPOP-normal}

\includegraphics[width=\textwidth]{figure-unconstrained-PDPA-normal}

\bibliographystyle{abbrvnat}
\bibliography{refs}

\end{document}

